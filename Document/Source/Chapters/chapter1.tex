\chapter{Introduction}
\section{Motivation}

An accurate detection and extraction of the moving objects in a video captured by a moving camera is a difficult task. 

Various approaches have been discussed in literature to address this problem statement.



Motion is one of the most important cue in the human visual perception used to sense the surrounding environment. Particularly, it is of great importance for interpreting visual information such as grouping objects and perceiving their structure, estimating speed and performing self-localization.

A common technique to exploit motion information is to make use of the concept of the optical flow.

ILLUSTRATE BY GIVING AN EXAMPLE: MOVING CAR



optical flow can be used to model motion and is one of the most dominant bottom-up cues in visual systems.


ILLUSTRATE BY GIVING AN EXAMPLE: MOVING CAR


By definition, motion is the change in position of an object with respect to time. 



Point trajectories are an intermediate representation for the motion of objects.

long-term motion is described by a spatio-temporal curve called trajectory


segmenting the moving objects in a video by analysing of its long-term point trajectories.

Detection of moving objects in sequences is an essential step for video analysis.

It  becomes  a  very  difficult  task  in  the  presence  of  camera  movement and  dynamic background

moving camera, and containing complex, and sometimes large, motions in
the background

use concept of motion to introduce the topic motion segmentation
motion a fundamental cue of visual perception.
example moving car


\section{Goals}

The purpose of this thesis is to describe a motion segmentation method that makes use of visual cues, such as the optical flows and depths. In addition, this thesis evaluates different flow-and segmentation methods and discusses their shortcomings.


implement a pipeline that can perform the task of motion segmentation using various visual cues.
evaluate differnt methods and find their drawbacks and when the peform the best

\section{Previous Work}
\section{Thesis Structure}