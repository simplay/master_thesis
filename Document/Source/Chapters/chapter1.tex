% On the other hand, assuming purely rigid motions is a strong restriction that is barely fulfilled in organic shapes. When a person moves, there are parts of their body moving rigidly (e.g. upper and lower arms or legs) and others which are transitions between the rigid ones (e.g. the neck). Be- sides, rigid motions within a fine-grained articulated struc- ture may not be observable with the limited resolution of a camera. For these reasons, a sharp segmentation will never be able to estimate the motion of life beings or some other inanimate objects with exactitude.


%TODO: think about where to explain the difference between rigid-and deformable motion.


% RELATED WORK - OF
%=> SRSF
%source 1:
% 
%propose    a    per-pixel    over-
%parameterized representation using rotation and translation
%to  capture  the  global  rigid  motion  of  the  scene  while
%allowing  local  deviations.    Their  method  is  more  suited
%for  a  single  moving  object.   By  comparison  our  method
%captures the global 3D motion for each individually moving
%object  and  can  handle  multiple  moving  objects
%source 2:
%
%overparametrize scene flow and estimate a 6-DoF rigid- body motion at each pixel. They regularize the flow field in this 6-DoF parametrization such that their model favors lo- cally rigid motions

%describe long-term motion of tracked points by a spatio-temporal curve called trajectory.
%method that generates such point trajectories for a given point sampling rated is presented. 
%based on LDOF all points showing some underlying structure are tracked until they are occluded. the occlusion reasoning is done based on the comparison of the forward and backward optical flow. when trajectroies get lost due to occlusion such that the sampling is sparser than the desidred rate, a new trajectory is started. the result is a set of reliable trajectories, that start in some frame of the sequence and end in another. depending on the data, many trajectories do not have any frame on common. However, the longer the are, the more useful motion information they are expected to carry. 
%
%top-down approach require ground truth prior information.
%thus, use a bottom-up approach.
%
%value of motion and gestalt principle of common fate
%motion vectors are typically more homogeneous within an object region than color and texture. consequently, ambiguities in color baed segmentation disappear as soon as objects move.
%
%however, most objects do not move permanently (sometimes a moving object stops moving). moreover, articulated objects do not move homogeneously. deformation of body parts.
%=> causes severe problems in typical motion segmentation approaches based on two-frame optical flow.
%tracking the interplay of the articulated parts over longer periods yields the missing information about the overall motion.
%=> hence, motion should be analyzed over longer periods (decreases the intra motion variance relative to the inter-object variance). moreover, motion information can be propagated to frames in which the object is mainly static.
%
%for such long term analysis, a tracking is required.
%point tracking is more reliable than superpixels, as stable features are located at edges and corners, rather than in flat areas.
%
%point trajectories are an informative intermediate representation for the motion of object parts or whole objects.
%
%use a semi-dense point tracker based on optical flow that yields reliable trajectories for hundreds of frames.
%eloberate
%
%due to occlusion and disocclusion, 
%trajectories are usually asynchronous, i.e. they start and end in different frames. 
%
%most existing trajectory clustering methods cannot deal with this problem. 
%
%therefore, we define a distance between all pairs of trajectories that share some common frames.
%
%sparse sampling of the frames with our trajectories, 
%we run spectral clustering on all extracted trajectories.
%
%
%long term aspect: motion is not considered independently for each frame but regard the whole motion history of a point to make a grouping decision.
%this requires point trajectories
%
%to increase the quality of the point tracking, especially at very fast moving regions: 
%
%track points based on an optical flow method (lodf). 
%
%video segmentation approach: based on spectral clustering, where eigenvectors of the normalized graph laplacian are used to generate segmentations that approximate the optimal normalized cut.
% Point trajectories are an intermediate representation for the motion of objects.
% long-term motion is described by a spatio-temporal curve called trajectory
% segmenting the moving objects in a video by analysing of its long-term point trajectories.

%sources
% optical flow example: http://cbia.fi.muni.cz/projects/optical-flow-for-live-cell-imaging_3.html
% motion segmentation example: http://lmb.informatik.uni-freiburg.de/research/segmentation/
% 
\chapter{Introduction}
\section{Motivation}
In nature one important aspect that determines the odds of survival and thus the individual success is the ability to sense the surrounding environment. In particular, the perceptional system of any advanced species is directed by visual cues amongst others. Alongside with sensing color, depths and brightness, the perception of motion is a fundamental visual cues. Particularly, is of great importance for interpreting visual information such as object groupings and their structure, estimating speed and performing self-localization. 
\begin{figure}[H]
\begin{center}
\subfigure[Frame 1]{
   \includegraphics[width=0.31\linewidth] {evaluation/bonn_chairs_frames/1}
   \label{fig:motivating_example_a}
}
\subfigure[Frame 12]{
   \includegraphics[width=0.31\linewidth] {evaluation/bonn_chairs_frames/12}
   \label{fig:motivating_example_b}
}
\subfigure[Frame 24]{
   \includegraphics[width=0.31\linewidth] {evaluation/bonn_chairs_frames/24}
   \label{fig:motivating_example_c}
}
\end{center}
\caption[Motivating Example]{foobar}
\label{fig:motivating_example}
\end{figure}
Especially the task of an accurate detection and extraction of moving objects in a video, captured by a moving camera, is nowadays still a difficult task. \\ \\
The field of motion segmentation aims at decomposing a video in moving objects and background and s the first fundamental step in many computer vision algorithms. It has many applications in the fields of robotics, metrology, video surveillance, traffic monitoring.
\begin{figure}[H]
\begin{center}
\subfigure[Optical Flow Field]{
   \includegraphics[width=0.47\linewidth] {introduction/motivation/motion_segmentation/optical_flow_visualization}
   \label{fig:motion_segmentation_motivation_eg_a}
}
\subfigure[Motion Segmentation]{
   \includegraphics[width=0.47\linewidth] {introduction/motivation/motion_segmentation/dense_motion_segmentation}
   \label{fig:motion_segmentation_motivation_eg_b}
}
\end{center}
\caption[Motion Segmentation Motivation Example]{foobar}
\label{fig:motion_segmentation_motivation_eg}
\end{figure}
A common technique to address such motion segmentation tasks is to make use of the optical flow, the apparent visual motion that a moving viewer experiences. \\ \\

\section{Goals}
The purpose of this thesis is to describe a robust motion segmentation method
that uses RGB-D$\footnote{The term \textit{RGB-D} refers to color images with associated depth field measurements.}$ video sequences as well as other visual cues. The actual problem statement we want to address is shown in figure $\ref{fig:problem_statement}$. 
\begin{figure}[H]
\begin{center}
\includegraphics[width=1.05\linewidth] {introduction/problem_statement_ref}
\end{center}
\caption[Problem Statement]{ Graphical representation of the problem statement we want to solve in this thesis. Given a video sequence, its corresponding depth fields and the used camera calibrations, we want to extract the frames locations that mask the moving objects via segmentation.}
\label{fig:problem_statement}
\end{figure}
For a given RGB-D video sequence and the camera calibration data, we want to detect and extract the individual moving objects. We therefore develop a pipeline that is capable of generating such motion segmentations by using optical flows. \\ \\
Moreover, we want to examine the influence of the utilized pipeline components. Hence, we integrate various segmentation-and flow-techniques into our implementation. We then quantitatively evaluate segmentations generated by all possible pipeline combinations. \\ \\
In summary, we want to achieve the following three goals:
\begin{itemize}
  \item Implement a pipeline that can segment moving rigid objects by running a long term analysis of point trajectories by using optical flows and depth cues.
  \item Examine the influence of the used flows, the method of trajectory analysis and segmentation and how they affect the segmentation quality by varying those components in the pipeline. For that purpose our pipeline implements various state of the art flow generation methods, different analysis approaches and a series of segmentation techniques.
  \item Quantitatively evaluate the resulting segmentation quality produced by running various pipeline combinations by comparing them against manually drawn ground truth segments.
\end{itemize}
Additionally, our implementation should be robust according to complex camera movement and noise, should be able to deal with occlusions and missing data and can handle many moving objects. \\ \\

\section{Related Work}
In this thesis we encounter and use many concepts from various fields. In particular we have deal with optical flow estimation methods, motion tracking analysis techniques and different segmentation concepts. In the following we give a brief summary of related work, state which parts are common with respect to our implementation and lastly, we state our contribution.

\subsection{Optical Flow}
list various existing methods to compute the optical flow

a vector motion field which describes the distribution of the apparent velocities of brightness patterns in a sequence. 

first formalized and computed  for  image  sequences  by  Horn  and  Schunck $\cite{Hs81}$ in  the  1980. Since the pioneering work of Horn and Schunck, many other approaches have been proposed:

\subsection{Motion Segmentation}
In this section we describe different approaches that can be used to address the task of motion segmentation.

\paragraph{Image Difference:} Taking the difference between two frames is a simple technique for detecting changes. By thresholding the intensity difference of two frames, a coarse map of the temporal changes can be obtained. Despite its simplicity, this techniques is limited by the presence of noise, camera movement (everything is changing) and low frame rates (nothing is changing). However, instead of directly using the difference image as the segmentation, the rough map of the changing areas can be used as a guide to extract spatial or temporal information in order to track the regions. Image difference techniques have been used in the work of $\cite{Cav05}$, $\cite{Li07}$ and $\cite{Col07}$.

\paragraph{Statistical Approach} Motion segmentation can be related to a classification problem, identifying whether a particular pixel belongs to either the background or the foreground. In $\cite{Cre05}$ the authors present a variational approach for segmenting an image into a set of regions of parametric motions using a model based on a conditional probability for the spatio-temporal image gradient. By exploiting Bayesian theory, they derive a cost functional which depends on parametric motion models for each  of a set of regions and on the boundary separating these regions.

\paragraph{Wavelets}
The idea is to exploit the ability of wavelets to perform analysis of the different frequency components of the images and then study each component with a resolution matched to its scale. Usually wavelet multi-scale decomposition is used in order to reduce the noise and in conjunction with other approaches, such as optical flow.

\paragraph{Layer Based} The key idea of layers based techniques is to understand which are the different depth layers in the image and which objects lie on which layer. 

\paragraph{Factorization Methods} Allow to recover the structure and motion by using traced features over a the whole image sequence.

\paragraph{Optical Flow}
track features using the optical flow over the given image sequence, compute affinity matrix based on similarity between the overlapping trajectories, use affinity matrix as an input to solve the segmentation task. LIST work but only shallow and briefly

$\cite{Bro11a}$

\section{Thesis Structure}

The reminder of this thesis is organized as follows:
In chapter 2 we start by providing the reader the relevant background in order to following the later chapters of this thesis. This chapter gives a detailed introduction to the concept of optical flow, to segmentation, spectral clustering and graph cut. It also explains how motion segmentation using optical flow is usually approached. \\ \\
In chapter 3 we develope all relevant derivations used to implemented the final pipeline. \\ \\
In chapter 4 we describe the implementation of our motion segmentation pipeline and all its stages in detail. \\ \\
In chapter 5 we list and discuss the results of some performed experiments. We start by introducing the reader to our used datasets. In our experiments section, we perform a qualitative-, as well as a quantitative evaluation. We explore the impact of the huge parameter-space due to the various stages of the implemented pipeline. Last, we also show some good segmentations, using the best (greedy found) parameters on our datasets. \\ \\
And finally Chapter 6 contains the conclusion of this thesis discussing what has been achieved in this thesis and the drawbacks of the proposed method. It also contains a note about some of my personal experiences during this thesis. 