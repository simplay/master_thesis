\chapter{Introduction}
\section{Motivation}

Motion describes the change in position of an object with respect to time and is one of the most fundamental cue in the human visual system used to perceive the surrounding environment. Particularly, it is of great importance for interpreting visual information such as object groupings and their structure, estimating speed and performing self-localization.

Especially the task of an accurate detection and extraction of moving objects in a video, captured by a moving camera, is nowadays still a difficult task. 
To address such tasks, a common technique is to exploit motion information by making use of the optical flow. The optical flow models the apparent visual motion that a moving viewer experiences.

ILLUSTRATE BY GIVING AN EXAMPLE: MOVING CAR

WIRTE MORE MOTIVATION

% Point trajectories are an intermediate representation for the motion of objects.
% long-term motion is described by a spatio-temporal curve called trajectory
% segmenting the moving objects in a video by analysing of its long-term point trajectories.

\section{Goals}

The purpose of this thesis is to describe a motion segmentation method that makes use of visual cues, such as the optical flows. In particular we want to exploit available depth information and illustrate how the quality of the results can be enhanced. In addition, this thesis evaluates different flow-and segmentation methods and discusses their shortcomings.

List of main contribution

\begin{itemize}
  \item Give an introduction into the field of optical flow, motion segmentation and clustering algorithms.
  \item implement a pipeline that can perform the task of motion segmentation using various visual cues.
  \item evaluate and compare the different approaches and find their drawbacks and benefits.
  \item Use the acquired knowledge and tweak the pipeline.
\end{itemize}


clearly formulate the problem statement
clearly formulate all assumptions

\section{Related Work}
initially, list its origin and then, list related contributions in the fields of optical flow, motion segmentation using optical flow and other motion segmentation approaches. 



$\cite{Bro11a}$

\section{Thesis Structure}

list all chapters and mention them by one sentence