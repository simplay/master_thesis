\chapter{Introduction}

describe long-term motion of tracked points by a spatio-temporal curve called trajectory.
method that generates such point trajectories for a given point sampling rated is presented. 
based on LDOF all points showing some underlying structure are tracked until they are occluded. the occlusion reasoning is done based on the comparison of the forward and backward optical flow. when trajectroies get lost due to occlusion such that the sampling is sparser than the desidred rate, a new trajectory is started. the result is a set of reliable trajectories, that start in some frame of the sequence and end in another. depending on the data, many trajectories do not have any frame on common. However, the longer the are, the more useful motion information they are expected to carry. 

top-down approach require ground truth prior information.
thus, use a bottom-up approach.

value of motion and gestalt principle of common fate
motion vectors are typically more homogeneous within an object region than color and texture. consequently, ambiguities in color baed segmentation disappear as soon as objects move.

however, most objects do not move permanently (sometimes a moving object stops moving). moreover, articulated objects do not move homogeneously. deformation of body parts.
=> causes severe problems in typical motion segmentation approaches based on two-frame optical flow.
tracking the interplay of the articulated parts over longer periods yields the missing information about the overall motion.
=> hence, motion should be analyzed over longer periods (decreases the intra motion variance relative to the inter-object variance). moreover, motion information can be propagated to frames in which the object is mainly static.

for such long term analysis, a tracking is required.
point tracking is more reliable than superpixels, as stable features are located at edges and corners, rather than in flat areas.

point trajectories are an informative intermediate representation for the motion of object parts or whole objects.

use a semi-dense point tracker based on optical flow that yields reliable trajectories for hundreds of frames.
eloberate

due to occlusion and disocclusion, 
trajectories are usually asynchronous, i.e. they start and end in different frames. 

most existing trajectory clustering methods cannot deal with this problem. 

therefore, we define a distance between all pairs of trajectories that share some common frames.

sparse sampling of the frames with our trajectories, 
we run spectral clustering on all extracted trajectories.


long term aspect: motion is not considered independently for each frame but regard the whole motion history of a point to make a grouping decision.
this requires point trajectories

to increase the quality of the point tracking, especially at very fast moving regions: 

track points based on an optical flow method (lodf). 

video segmentation approach: based on spectral clustering, where eigenvectors of the normalized graph laplacian are used to generate segmentations that approximate the optimal normalized cut.


\section{Motivation}

Motion describes the change in position of an object with respect to time and is one of the most fundamental cue in the human visual system for perceiving the surrounding environment. Particularly, it is of great importance for interpreting visual information such as object groupings and their structure, estimating speed and performing self-localization.

Especially the task of an accurate detection and extraction of moving objects in a video, captured by a moving camera, is nowadays still a difficult task. A common technique to address such motion segmentation tasks is to make use of the optical flow, the apparent visual motion that a moving viewer experiences. 



Motion segmentation aims at decomposing a video in moving objects and background.
Motion segmentation is the first fundamental step in many computer vision algorithms.
Motion segmentation applications: robotics, metrology, video surveillance, traffic monitoring


MENTION THAT IT IS USED IN NATURE
MENTION MOTION discontinuities, which CAN BE exploited as motion object boundaries.


ILLUSTRATE BY GIVING AN EXAMPLE: MOVING CAR

WIRTE MORE MOTIVATION

% Point trajectories are an intermediate representation for the motion of objects.
% long-term motion is described by a spatio-temporal curve called trajectory
% segmenting the moving objects in a video by analysing of its long-term point trajectories.

\section{Goals}

The purpose of this thesis is to describe a motion segmentation method that makes use of visual cues, such as the optical flows. Moreoever, we want to exploit available depth information and illustrate how the quality of the results can be enhanced. In particular, the presented method should fulfill the follwing constraints:

\begin{itemize}
  \item supports a moving camera
  \item be able to deal with occlusions
  \item can handle multiple objects
  \item is robust to nois
  \item can process a coherent sequence of images (video)
  \item can deal with missing data
\end{itemize}

assumptions: only handles rigid motion
 
In addition, this thesis evaluates different flow-and segmentation methods and discusses their shortcomings.

List of main contribution

\begin{itemize}
  \item Give an introduction into the field of optical flow, motion segmentation and clustering algorithms.
  \item implement a pipeline that can perform the task of motion segmentation using various visual cues.
  \item evaluate and compare the different approaches and find their drawbacks and benefits.
  \item Use the acquired knowledge and tweak the pipeline.
\end{itemize}


clearly formulate the problem statement
clearly formulate all assumptions

\section{Related Work}

\subsection{Optical Flow}
list various existing methods to compute the optical flow

a vector motion field which describes the distribution of the apparent velocities of brightness patterns in a sequence. 

first formalized and computed  for  image  sequences  by  Horn  and  Schunck $\cite{HS}$ in  the  1980. Since the pioneering work of Horn and Schunck, many other approaches have been proposed:

\subsection{Motion Segmentation}
In this section we describe different approaches that can be used to address the task of motion segmentation.

\paragraph{Image Difference:} Taking the difference between two frames is a simple technique for detecting changes. By thresholding the intensity difference of two frames, a coarse map of the temporal changes can be obtained. Despite its simplicity, this techniques is limited by the presence of noise, camera movement (everything is changing) and low frame rates (nothing is changing). However, instead of directly using the difference image as the segmentation, the rough map of the changing areas can be used as a guide to extract spatial or temporal information in order to track the regions. Image difference techniques have been used in the work of $\cite{Cav05}$, $\cite{Li07}$ and $\cite{Col07}$.

\paragraph{Statistical Approach} Motion segmentation can be related to a classification problem, identifying whether a particular pixel belongs to either the background or the foreground. In $\cite{Cre05}$ the authors present a variational approach for segmenting an image into a set of regions of parametric motions using a model based on a conditional probability for the spatio-temporal image gradient. By exploiting Bayesian theory, they derive a cost functional which depends on parametric motion models for each  of a set of regions and on the boundary separating these regions.

\paragraph{Wavelets}
The idea is to exploit the ability of wavelets to perform analysis of the different frequency components of the images and then study each component with a resolution matched to its scale. Usually wavelet multi-scale decomposition is used in order to reduce the noise and in conjunction with other approaches, such as optical flow.

\paragraph{Layer Based} The key idea of layers based techniques is to understand which are the different depth layers in the image and which objects lie on which layer. 

\paragraph{Factorization Methods} Allow to recover the structure and motion by using traced features over a the whole image sequence.

\paragraph{Optical Flow}
track features using the optical flow over the given image sequence, compute affinity matrix based on similarity between the overlapping trajectories, use affinity matrix as an input to solve the segmentation task. LIST work but only shallow and briefly

$\cite{Bro11a}$

\section{Thesis Structure}

The reminder of this thesis is organized as follows:
In chapter 2 we start by providing the reader the relevant background in order to following the later chapters of this thesis. This chapter gives a detailed introduction to the concept of optical flow, to segmentation, spectral clustering and graph cut. It also explains how motion segmentation using optical flow is usually approached. \\ \\
In chapter 3 we develope all relevant derivations used to implemented the final pipeline. \\ \\
In chapter 4 we describe the implementation of our motion segmentation pipeline and all its stages in detail. \\ \\
In chapter 5 we list and discuss the results of some performed experiments. We start by introducing the reader to our used datasets. In our experiments section, we perform a qualitative-, as well as a quantitative evaluation. We explore the impact of the huge parameter-space due to the various stages of the implemented pipeline. Last, we also show some good segmentations, using the best (greedy found) parameters on our datasets. \\ \\
And finally Chapter 6 contains the conclusion of this thesis discussing what has been achieved in this thesis and the drawbacks of the proposed method. It also contains a note about some of my personal experiences during this thesis. 