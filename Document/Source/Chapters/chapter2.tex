\chapter{Theoretical Background}

\section{Optical Flow}
\begin{itemize}
  \item What is Optical Flow
  \subitem Example
  \subitem Definition
  \item Original Approach to Compute it.
\end{itemize}


visual phenomenon that describes the perception of motion.

apparent visual motion a moving viewer experiences.
assume: given a moving viewer, looking at an object, then, 

objects passing by appear moving backwards
distant objects seem to move very slowly and close objects appear to move fast.
relationship between the optical flow and the viewed object. The magnitude of the optical flow is doubles if either

the speed of the traveling viewer is doubled or the distance to the observed object is halved.

the optical flow varies depending on the angle between the viewing direction and and the direction of movement of the observed object.

suppose the viewer travels forward, then, the optical flow is fastest if either the object is moving orthogonally towards the viewer's direction or it is moving directly above, below to the 

an object directly in front of a viewer will not have any optical flow and thus appear to stand still. However, since not the whole silhouette of the object is in front of the viewer, its edges will appear to move and thus the objects appears to get larger. 
 

\subsection{Determining Optical Flow - Berthold K.P. Horn and Brian G. Schnunck}

Is the distribution of apparent velocities of movement of brightness patterns in an image.
Can arise from relative motion of objects and the viewer.
Can give information about the spatial arrangement of the viewed objects and the rate of change in that arrangement.
Discontinuities in the optical flow can help to perform segmentation tasks.

the optical flow cannot be computed a point in the image independently of neighboring points without introducing additional constraints

reason: the velocity field any image point has two components while its change in image brightness due to motion yields only one constraint.

their example: consider a patch of a pattern where brightness varies as a function of one image coordinate, but not the other. Movement of the pattern in one direction alters the brightness at a particular point, but motion in the other direction yields no change. Thus components of movement in the latter direction cannot be determined locally.

their problem statement:
they initially assume that 

1. the surface is flat to avoid brightness variations due to shading effects.
2. the incident illumination is uniform across the surface. Then, the brightness at a point in the image is proportional to the reflectance of the surface at the corresponding point on the object.
3. reflectance varies smoothly and has no spatial discontinuities. Having no discontinuities assures that the image brightness is differentiable.
4. Situations where objects occlude one another are excluded.

derive equation that relates the change in image brightness at a point to the motion of the brightness pattern.

point $p(x,y)$, time $t$, brightness at p in the image plane at t $E(x,y,t)$.
When the pattern moves, the brightness of a particular point in the pattern is constant. 
$\frac{d E}{dt} = 0 <=> \frac{\partial E}{\partial x} \frac{x}{dt} + \frac{\partial E}{\partial y} \frac{y}{dt} + \frac{\partial E}{\partial t} = 0 <=> E_{x} u + E_{y} v + E_{t} = 0$.

where $u = \frac{dx}{dt}$ and $v = \frac{dy}{dt}$. Hence, we have a single linear equation with two unknowns $u$ and $v$.

solution: introduce a smoothness constraint.


\section{Motion Segmentation}
motion segmentation aims at decomposing a video in moving objects and background by segmenting the objects that undergo different motion patterns.
\section{Spectral Clustering}
\section{Graph Cut}
\section{Kernighan-Lin}
