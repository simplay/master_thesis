\chapter{Theoretical Background}

\section{Optical Flow}

Determining Optical Flow - Berthold K.P. Horn and Brian G. Schnunck

Is the distribution of apparent velocities of movement of brightness patterns in an image.
Can arise from relative motion of objects and the viewer.
Can give information about the spatial arrangement of the viewed objects and the rate of change in that arrangement.
Discontinuities in the optical flow can help to perform segmentation tasks.

the optical flow cannot be computed a point in the image independently of neighboring points without introducing additional constraints

reason: the velocity field any image point has two components while its change in image brightness due to motion yields only one constraint.

their example: consider a patch of a pattern where brightness varies as a function of one image coordinate, but not the other. Movement of the pattern in one direction alters the brightness at a particular point, but motion in the other direction yields no change. Thus components of movement in the latter direction cannot be determined locally.

their problem statement:
they initially assume that 

1. the surface is flat to avoid brightness variations due to shading effects.
2. the incident illumination is uniform across the surface. Then, the brightness at a point in the image is proportional to the reflectance of the surface at the corresponding point on the object.
3. reflectance varies smoothly and has no spatial discontinuities. Having no discontinuities assures that the image brightness is differentiable.
4. Situations where objects occlude one another are excluded.

derive equation that relates the change in image brightness at a point to the motion of the brightness pattern.

point $p(x,y)$, time $t$, brightness at p in the image plane at t $E(x,y,t)$.
When the pattern moves, the brightness of a particular point in the pattern is constant. 
$\frac{d E}{dt} = 0 <=> \frac{\partial E}{\partial x} \frac{x}{dt} + \frac{\partial E}{\partial y} \frac{y}{dt} + \frac{\partial E}{\partial t} = 0 <=> E_{x} u + E_{y} v + E_{t} = 0$.

where $u = \frac{dx}{dt}$ and $v = \frac{dy}{dt}$. Hence, we have a single linear equation with two unknowns $u$ and $v$.

solution: introduce a smoothness constraint.

