\chapter{Implementation}
In this chapter we explain in detail the relevant stages of our motion segmentation pipeline. Moreover, for each stage, we describe its required input data, what output it generates and all its assumptions and induced limitations. 

problem statement: find the moving object of a video
input: dataset, depth images
output: segmentation and pipeline dumps


mention pipeline assumptions
list and explain different pipeline stages:

\begin{enumerate}
\item Preprocessing
	\subitem Filter input images
	\subitem Rename images to satisfy pipeline naming conventions 
\item Compute optical flow
	\subitem Four different methods are available:
	\subitem Large displacement optical flow (LDOF)
	\subitem Original flow method proposed by Horn and Schunck (HS)
	\subitem Semi-rigid scene flow (SRSF)
	\subitem Layered RGBD flow (LRGB)
\item Extracting relevant data
	\subitem traceable candidates
	\subitem depth fields
	\subitem color values
	\subitem flow-and depth field variances.
\item Compute trajectories and their affinities
\item Compute sparse segmentation
	\subitem There are three different segmentation methods available
	\subitem Spectral clustering of the affinity matrix
	\subitem MinCut
	\subitem Kernighan-Lin 
\item Compute dense segmentation
	\subitem use the dense segmentation as an input solve the problem of whole filling using a primal-dual convex algorithm. 
\item Evaluate quality
	\subitem statistically evaluate the quality of the  
\end{enumerate}

discuss the details about each pipeline stage
mention in which language the individual parts are coded.
