\chapter{Implementation}
In this chapter we explain in detail the relevant stages of our motion segmentation pipeline. We start by formulate the initial problem statement our pipeline tries to solve. Moreover, for each stage, we describe its required input data and what output it generates. Lastly, we also mention all assumptions and induced limitations for each pipeline stage. 

The problem statement our pipeline solves is defined as follows: 
Given a set of images that form a video sequence and their associated depth maps. Our goal is to separate the images into regions that form coherent and independent rigidly moving objects. 

%SHOW FIGURE OF PROBLEM STATEMENT
% INPUT: IMGS, DEPTHS, OUTPUT: SEGMENTATION
% SHOW LATER A DETAILED PIPELINE

The idea of such a segmentation is to identify and extract the meaningful rigid motions from the background. To accomplish this task we compute the optical flow on the images and use it as a guide for grouping pixel regions that belong together. 





list and explain different pipeline stages:

\begin{enumerate}
\item Preprocessing
	\subitem Filter input images
	\subitem Rename images to satisfy pipeline naming conventions 
\item Compute optical flow
	\subitem Four different methods are available:
	\subitem Large displacement optical flow (LDOF)
	\subitem Original flow method proposed by Horn and Schunck (HS)
	\subitem Semi-rigid scene flow (SRSF)
	\subitem Layered RGBD flow (LRGB)
\item Extracting relevant data
	\subitem traceable candidates
	\subitem depth fields
	\subitem color values
	\subitem flow-and depth field variances.
\item Compute trajectories and their affinities
\item Compute sparse segmentation
	\subitem There are three different segmentation methods available
	\subitem Spectral clustering of the affinity matrix
	\subitem MinCut
	\subitem Kernighan-Lin 
\item Compute dense segmentation
	\subitem use the dense segmentation as an input solve the problem of whole filling using a primal-dual convex algorithm. 
\item Evaluate quality
	\subitem statistically evaluate the quality of the  
\end{enumerate}

mention pipeline assumptions
discuss the details about each pipeline stage
mention in which language the individual parts are coded.

\subsection{Precomputations}
Refinement of input data
Optionally generate filtered version of input data
re-enumerate image to have a well defined naming

\subsection{Computing Optical Flows}

Optical flows are used to trace coherent tacking points that form a trajectory.

\subsection{Extracting Core Data}
\subsection{Compute Trajectories and Affinity Matrix}
\subsection{Sparse Motion Segmentation}
\subsection{Dense Motion Segmentation}
\subsection{Evaluation Program}
