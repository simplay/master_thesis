\chapter{Implementation}
In this chapter we explain in detail the relevant stages of our motion segmentation pipeline. We start by formulate the initial problem statement our pipeline tries to solve. Moreover, for each stage, we describe its required input data and what output it generates. Lastly, we also mention all assumptions and induced limitations for each pipeline stage. \\ \\
The problem statement our pipeline solves is defined as follows: 
Given a set of images that form a video sequence and their associated depth maps. Our goal is to separate the images into regions that form coherent and independent rigidly moving objects. \\ \\
%SHOW FIGURE OF PROBLEM STATEMENT
% INPUT: IMGS, DEPTHS, OUTPUT: SEGMENTATION
% SHOW LATER A DETAILED PIPELINE
The idea of such a segmentation is to identify and extract the meaningful rigid motions from the background. To accomplish this task we compute the optical flow on the images and use it as a guide for grouping pixel regions that belong together. Since the object motion in coherent image sequence is not independent for every frame, the optical flow depicts an ideal cue for making grouping decisions. \\ \\
Our pipeline has the following main stages:

\begin{enumerate}
\item \textbf{Dataset Preparations}: Initially, the frames of a video have to be extracted and named according to the pipeline naming conventions. If present, also the depth maps have to be named and re-normalized according to the pipeline's conventions. Optionally, a blurred version of the input images are computed by using a bilateral filter. This filtered images act as an additional, but optional cue, for a later pipeline stage. 
\item \textbf{Optical Flows}: We compute the forward- and backward-flows on our input sequence using different existing implementations. Our pipeline lets the user select a target method for generating the optical flow fields.
\item \textbf{Data Extraction}: In this stage we extract traceable feature locations in our images. Also, a mask containing all invalid tracking locations per image is computed by checking whether the the forward and backward flow correspond to each other. Optionally, depth data, flow and depth variances and color maps are extracted that are used within a later pipeline stage. 
\item \textbf{Trajectories}: We use the computed forward flows and the previously extracted traceable feature locations to perform a point tracking. A sequence of traced points form a so called trajectory. In every frame a trajectory can either be started, ended or be continued. The previously computed traceable locations act as the starting points, the forward flow is added to a traceable location and yield the tracked to position, the starting location in the next frame. Tracked to locations that land in an invalid mask location cause the tracking of a trajectory. 
\item \textbf{Affinity Matrix}: In this stage we compute the similarities between all trajectory pairs. The similarity is computed according to different metrics, which are basically a combination of various distances among overlapping trajectory parts. 
\item \textbf{Sparse Motion Segmentation}: Using the affinity matrix plus the nearest neighbouring trajectories per trajectory, we can compute its dense segmentation by either applying a spectral clustering on the affinity matrix or by reformulating the problem as a graph cut problem. Usually 
\item \textbf{Dense Segmentation}: Our pipeline allows to transform the sparse segmentation into a dense segmentation. For this purpose we consider the hole filling problem, formulating it as a convex problem and then solving it using a primal-dual approach using the sparse segmentation as the initial input.
\item \textbf{Evaluation}: Our pipeline allows to qualitatively and quantitatively evaluate the generated sparse and dense segmentations. In addition, we also can explore various parameter settings and their outcomes using different visualizers.
\end{enumerate}
In the following sections we examine and discuss each individual pipeline stage in detail. 

\section{Dataset Preparations}
The preparation of an input datasets is the very first stage of our pipeline. Initially, we have to either capture a video using one of our capturing devices or use an existing video sequence. Next, we extract all the frames from the considered video. Optionally, there are also captured depth maps$\footnote{In case we are working with depth data, we assume that there exists one depth map per frame. Our pipeline assumes, that the values in the depth images are in meter units}$ available. For further information about the dataset conventions, read figure~\ref{sec:datasets} on page~\pageref{sec:datasets}.

Refinement of input data
Optionally generate filtered version of input data
re-enumerate image to have a well defined naming

Filter input images
Rename images to satisfy pipeline naming conventions

\section{Computing Optical Flows}

Optical flows are used to trace coherent tacking points that form a trajectory.

Four different methods are available:
Large displacement optical flow (LDOF)
Original flow method proposed by Horn and Schunck (HS)
Semi-rigid scene flow (SRSF)
Layered RGBD flow (LRGB)


\section{Extracting Core Data}


traceable candidates
depth fields
color values
flow-and depth field variances.

\section{Compute Trajectories and Affinity Matrix}
\section{Sparse Motion Segmentation}
There are three different segmentation methods available
Spectral clustering of the affinity matrix
MinCut
Kernighan-Lin 
\section{Dense Motion Segmentation}
use the dense segmentation as an input solve the problem of whole filling using a primal-dual convex algorithm. 
\section{Evaluation Program}
statistically evaluate the quality






mention pipeline assumptions
discuss the details about each pipeline stage
mention in which language the individual parts are coded.
