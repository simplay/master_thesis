\chapter{Conclusion}
\section{Summary}
The aim of this thesis was to develop a pipeline that can produce motion segmentations on RGBD-videos using optical flow fields. Based on various state-of-the-art contributions in the field of optical flow, clustering and motion tracking, we formulated a general motion segmentation pipeline. Such a pipeline consists of the following stages: estimating optical flow fields, a trajectory tracking using flow fields and a spectral clustering on the trajectories with respect to a specific distance measure. In particular our pipeline consists of a series of flow estimation methods, different similarity measures and multiple segmentation techniques. \\ \\ 
In our experiments we successfully demonstrated capability of our pipeline to produce good motion segmentations. Moreover, we statistically evaluated the quality of all possible pipeline combinations on various datasets by comparing the resulting segmentations against manually generated ground truth images. This way we were able to determine the strengths and weaknesses of different techniques and find an optimal setting. \\ \\
In particular, we were able to show quantitatively that incorporating depth cues into the individual pipeline components drastically improves the quality of the resulting motion segmentations. Moreover, we observed that the quality of the utilized flow fields significantly affects the segmentation quality. Hence, we concluded, that when using P-affinities, our best pipeline setup is $\text{SRSF PED MC}$ and when using S-affinities $\textit{SRSF SED KL}$. Both variants achieve best results compared against the other available setups. However, SRSF SED KL achieves better F1 scores but SRSF PED MC runs much faster. Therefore, we denote both methods as winners.  \\ \\
Moreover, when using our best pipeline setups, we could deal with complex scenes, which exhibit camera shaking, rotational movements and even slight non-rigid motions. Additionally, our pipeline is capable of dealing with noise in the flow fields by using various filters. Therefore, we conclude that we were able to fulfill our initially stated goals. \\ \\
The main contribution of this thesis are: the implementation of a motion segmentation pipeline, studying the influence on the motion segmentation quality when using different flow methods, similarity measures and segmentation techniques and a proof of concept that incorporating depth data into a motion segmentation pipeline yields better results than without. \\ \\
Regarding the limitations of this work we can state the following main issues: Our implementation is an over-paramterized framework. Many parameter values are chosen according to simple heuristics. For instance, the number of moving objects we want to extract is assigned manually for each dataset. The same holds true for the number of trajectory neighbors used by our segmentation methods, the number of eigenvectors and the values of $\lambda$. Another issue is the runtime of several pipeline components. For example, our KL implementation has a very poor runtime. Also, our segment merger method requires ground truth segmentation images images. However, relying on a sophisticated merging method as described in $\cite{OB14b}$ would be beneficial, since it does not assume any ground truth images. Lastly, we could adapt the pairwise affinity computation model by replacing it by a higher order model, similar as described in $\cite{OB12}$. \\ \\
Regarding future work we could think of many useful extensions and enhancements that directly address the stated limitations. The most beneficial extension, however, would be to automatically determine the optimal parameter setup using a learning based approach. Moreover, the pipeline could be ported to a GPU implementation to achieve interactive rates. We also could think of refining the flow estimation$\footnote{The quality of the optical flow fields strongly influences the resulting segmentations}$ by combining LDOF and SRSF. Such a method would be capable of dealing with large motions and use depths to deal with rotational movements and deformation. Lastly, we could think of developing a special neural network, which is capable of estimating depth maps by using a single color image. This idea is based on the work from paper $\cite{DBLP:journals/corr/EigenPF14}$.  

\section{Personal Experience}
After having successfully written my bachelor thesis at the Computer Graphics Group at the University of Bern, I was certain that, eventually, I would like to write my master thesis in a related field of Computer Graphics and Vision. Since I am generally very interested in topics like numerics and optimization problems, I asked for a thesis subject that combines Mathematics with practical problem. I have to admit that, after having read papers about segmentation techniques and variational methods at the very beginning of my thesis, I thought this topic would exceed my knowledge. And it actually did, but I decided not to give up. \\ \\
During working on my thesis I learnt many new concepts in Mathematics relating to optical flow, motion segmentation, clustering and graph partitioning methods, solving optimization problems and deepened my knowledge of numerics. I felt it as a satisfactory experience to use and apply all the knowledge which I have acquired during the time as a bachelor and master student. Furthermore, this thesis gave me the opportunity to program quite a lot. Hence I could strengthen my programming skills in Java and C++, Matlab and Ruby. Altogether it was a rewarding experience for me to write a Master thesis at the Computer Graphics Group.
\section{Acknowledgment}
First, I would like to thank Mr. Zwicker for giving me the opportunity to write a Master thesis at the Computer Graphics Group at the University of Bern. \\ \\
Foremost, I would like to express my sincere gratitude to my advisor Mr. Peter Bertholet for his continuous support of my study, his patience, motivation, enthusiasm, and knowledge. His guidance and active support helped me quite a lot while deriving the necessary formulations used by our pipeline, implementing them and evaluating the motion segmentations. \\ \\
Last but not least, I would like to thank my mother, Manuela Single and my brother Patrik Single and also to my close friend, Radischa Iyadurai for supporting me morally throughout during this thesis.